% Emacs, this is -*-latex-*-

\title{Ping}

\maketitle
\tableofcontents

\subsection{Measurement}
Ping can provide statistics about latencies.

\begin{figure}
  \begin{center}
    \myfig{graphics/ping_timeline}{8cm}{500}
    % \svgfig{graphics/ping_timeline}{8cm}{1800}
  \end{center}
  \caption{Timeline of a ping interaction between two hosts A and B,
    interconnected by simple communication link.}
  \label{fig:ping_timeline}
\end{figure}

To measure latencies, we will use
\href{https://github.com/torvalds/linux/blob/master/net/ipv4/ping.c}{\texttt{ping}}~\cite{Kurose-Ross,Forouzan},
a tool that
\href{https://en.wikipedia.org/wiki/Ping_(networking_utility)}{sends}
(one or more)
\href{https://en.wikipedia.org/wiki/Internet_Control_Message_Protocol}{ICMP}
Echo Request messages to an IP address and waits for receiving (one or
more) ICMP Echo Reply messages generated by the
(\href{https://en.wikipedia.org/wiki/Operating_system}{OS} of that)
host, measuring the so called
\href{https://en.wikipedia.org/wiki/Round-trip_delay}{RTT} (Round-Trip
Time). For example, in the Figure~\ref{fig:ping_timeline} are
described the different time components in which a RTT can be
decomposed. In this figure, $t_t$ stands for \emph{transmission time},
and $t_p$ (again) for \emph{propagation time}. A simple link (two
wires, for example) using
\href{https://en.wikipedia.org/wiki/Time-division_multiple_access}{TDM}
(Time-Domain Multiplexing) has been supposed. For this reason, the
propagation and transmission times are identical in both
directions. Notice that if the payload of the \verb|ping| message has
only 64 bytes (the default value in most \verb|ping| implementations)
and the bit rate of the link is high, then $t_p\gg t_t.$ In a LAN, for
example, it also holds that
\begin{equation}
  \text{RTT} = 2t_p + 2t_t.
  \label{eq:RTT}
\end{equation}

\bibliography{networking}
