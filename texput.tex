% Emacs, this is -*-latex-*-

\title{Ping}

\maketitle
\tableofcontents

\section{Time measurement}
Ping can provide statistics about latencies.

\begin{figure}
  \begin{center}
    \myfig{graphics/ping_timeline}{8cm}{500}
    % \svgfig{graphics/ping_timeline}{8cm}{1800}
  \end{center}
  \caption{Timeline of a ping interaction between two hosts A and B,
    interconnected by simple communication link.}
  \label{fig:ping_timeline}
\end{figure}

To measure latencies, we will use
\href{https://github.com/torvalds/linux/blob/master/net/ipv4/ping.c}{\texttt{ping}}~\cite{Kurose-Ross,Forouzan},
a tool that
\href{https://en.wikipedia.org/wiki/Ping_(networking_utility)}{sends}
(one or more)
\href{https://en.wikipedia.org/wiki/Internet_Control_Message_Protocol}{ICMP}
Echo Request messages to an IP address and waits for receiving (one or
more) ICMP Echo Reply messages generated by the
(\href{https://en.wikipedia.org/wiki/Operating_system}{OS} of that)
host, measuring the so called
\href{https://en.wikipedia.org/wiki/Round-trip_delay}{RTT} (Round-Trip
Time). For example, in the Figure~\ref{fig:ping_timeline} are
described the different time components in which a RTT can be
decomposed. In this figure, $t_t$ stands for \emph{transmission time},
and $t_p$ (again) for \emph{propagation time}. A simple link (two
wires, for example) using
\href{https://en.wikipedia.org/wiki/Time-division_multiple_access}{TDM}
(Time-Domain Multiplexing) has been supposed. For this reason, the
propagation and transmission times are identical in both
directions. Notice that if the payload of the \verb|ping| message has
only 64 bytes (the default value in most \verb|ping| implementations)
and the bit rate of the link is high, then $t_p\gg t_t.$ In a LAN, for
example, it also holds that
\begin{equation}
  \text{RTT} = 2t_p + 2t_t.
  \label{eq:RTT}
\end{equation}

\section{Bit-rate measurement}

Usually, to determine the throughput, we need to use a tool such as
\href{https://iperf.fr/}{iPerf} to
\href{https://en.wikipedia.org/wiki/Measuring_network_throughput}{measure
  the link capacity} between two
\href{https://datatracker.ietf.org/doc/html/rfc4113}{end-points} (of
different \href{https://en.wikipedia.org/wiki/Host_(network)}{hosts})
in the Internet. Unfortunately, this procedure implies that we must
have access to both hosts to install and run this program, privileges
that are not always available.

Alternatively, we can estimate the
\href{https://en.wikipedia.org/wiki/Throughput}{link throughput} using
\href{https://github.com/torvalds/linux/blob/master/net/ipv4/ping.c}{\texttt{ping}}. However,
it is important to realize that \texttt{ping} has been designed to
measure latencies, not bit-rates, and that, for this reason, we will
only be able to estimate throughputs, rougtly.

Said that, the transmission bitrate available between two directly
reachable IP devices can be found with
\begin{equation}
  b=\frac{B}{t_t},
  \label{eq:b}
\end{equation}
where $B$ is the number of bits sent in a \verb|ping|
\href{https://en.wikipedia.org/wiki/Payload_(computing)}{payload}, and
\begin{equation}
  t_t = \frac{\text{RTT}_{\text{max}}-2t_p}{2},
  \label{eq:tt}
\end{equation}
an equation that can be determined by revisiting the \verb|ping|
timeline shown in the Figure~\ref{fig:ping_timeline}
and the Eq.~\ref{eq:RTT}. In this expression, $\text{RTT}_{\text{max}}$ represents
  the average\footnote{Remember that ping can generate a number of
    time measurements.} RTT measured when we send the
  maximum\footnote{We should use a payload as large as possible
    (ideally, an infinite number of bytes), in order to make $t_t$ a
    significant amount of time, which is easy to measure} number of
  bytes in the payload of the \verb|ping|, and
\begin{equation}
  t_p \approx \frac{\text{RTT}_{\text{min}}}{2},
  \label{eq:tp}
\end{equation}
where $\text{RTT}_{\text{min}}$ presents the minimum\footnote{Again,
  remember that ping returns a collection of RTTs, a we should use the
  minimun beause the latency of the link should be a constant and the
  congestion always increases this value.} RTT measurement when we
send the minimum\footnote{If the payload of the ping is small enough
  ($0$ bytes ideally), it would happen that $t_t=0$.} payload.

Therefore, to estimate $b$ for the near future, we must perform the
following steps:

\begin{enumerate}
  
\item Run \texttt{ping} using the smallest possible payload that
  provides RTTs values.\footnote{The minimal payload for a
    \texttt{ping} message can be 0, but depending on the
    implementation of \texttt{ping} you might have to use a larger
    value.} Use it to calculate $t_p$.
  
\item Run \texttt{ping} using the largest possible payload to achieve
  that the most part of the average RTT provided by \texttt{ping}
  corresponds to $t_t$. It is necessary to use the highest
  payload-size because the accuracy of the cronometer used by
  \texttt{ping}, in general, is not high enough.
  
\item Use Eq.~\eqref{eq:tt} to find $t_t$ as a function of the average
  RTT and the minimal $t_p$.
  
\item Finally, use Eq.~\eqref{eq:b} to determine $b$.
  
\end{enumerate}

If the devices cannot be \texttt{ping}-ed, and supposing that there is
only one point in the communication link that is filtering the ICMP
Echo Request traffic in each direction, then we can compute the total RTT as
the sum of the RTTs from the devices that we want to connect (where
the InterCom is it supposed to be run) to the filter (possiblely, one
of your home routers).

You can use \href{https://www.meter.net/}{\texttt{www.meter.net}} to test your pings.

\bibliography{networking}
